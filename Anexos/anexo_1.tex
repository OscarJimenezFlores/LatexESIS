% ------------------- ANEXO 1 -------------------

\chapter*{Anexo 1. Matriz de consistencia}


\begin{longtable}{|p{4cm}|p{4cm}|p{4cm}|p{4cm}|p{4cm}|}
\hline
\textbf{Objetivos} & \textbf{Preguntas de Investigación} & \textbf{Hipótesis} & \textbf{Variables} & \textbf{Instrumentos de Recolección de Datos} \\
\hline
\endfirsthead  % Cabezal para la primera página

\hline
\textbf{Objetivos} & \textbf{Preguntas de Investigación} & \textbf{Hipótesis} & \textbf{Variables} & \textbf{Instrumentos de Recolección de Datos} \\
\hline
\endhead  % Cabezal para las páginas siguientes

\hline
\endfoot  % Pie de tabla al final

\hline
\endlastfoot  % Pie de tabla para la última página

% Ejemplo de una fila
\textbf{Objetivo General:} Determinar la influencia del Modelo TAM en la adopción de tecnologías en las organizaciones. \newline
\textbf{Objetivos Específicos:} 1. Evaluar la relación entre facilidad de uso percibida y adopción. \newline
2. Analizar la relación entre la utilidad percibida y adopción. \newline
3. Examinar el impacto de la actitud hacia la tecnología en la adopción. & 
\textbf{Pregunta General:} ¿Cómo influye el Modelo TAM en la adopción de tecnologías? \newline
\textbf{Preguntas Específicas:} 1. ¿La facilidad de uso percibida afecta la adopción de tecnología? \newline
2. ¿La utilidad percibida influye en la adopción de la tecnología? \newline
3. ¿La actitud hacia el uso impacta la adopción tecnológica? & 
\textbf{Hipótesis General:} El Modelo TAM influye positivamente en la adopción de tecnologías. \newline
\textbf{Hipótesis Específicas:} 1. La facilidad de uso percibida tiene un efecto positivo en la adopción. \newline
2. La utilidad percibida tiene un efecto positivo en la adopción. \newline
3. La actitud hacia la tecnología tiene un efecto positivo en la adopción. & 
\textbf{Variables Independientes:} 1. Facilidad de uso percibida. \newline
2. Utilidad percibida. \newline
3. Actitud hacia la tecnología. \newline
\textbf{Variable Dependiente:} Adopción de tecnologías. & 
\textbf{Instrumento:} Cuestionario estructurado con escalas de Likert. \newline
\textbf{Técnica de recolección:} Encuesta. \newline
\textbf{Población:} 200 empleados en empresas tecnológicas. \newline
\textbf{Muestra:} 120 empleados. \newline
\textbf{Método de Análisis:} Análisis estadístico de correlación. \\
\hline
\end{longtable}
