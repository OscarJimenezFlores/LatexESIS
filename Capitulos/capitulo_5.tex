% ------------------- CAPÍTULO V. ASPECTOS ADMINISTRATIVOS -------------------

% ------------------- CRONOGRAMA -------------------
\section{Cronograma de actividades}

A continuación, se presenta el cronograma de actividades planificadas para el desarrollo de la presente investigación. Las actividades se distribuyen a lo largo de seis meses, considerando las fases de planificación, recolección de datos, análisis y redacción final.

\begin{table}[H]
\centering
\begin{threeparttable}
\caption{Cronograma de actividades del proyecto de investigación}
\label{tab:cronograma}
\begin{tabular}{@{}p{7cm}cccccc@{}}
\toprule
\textbf{Actividad} & \textbf{Mes 1} & \textbf{Mes 2} & \textbf{Mes 3} & \textbf{Mes 4} & \textbf{Mes 5} & \textbf{Mes 6} \\
\midrule
Revisión de literatura & X & X &   &   &   &   \\
Diseño metodológico & X &   &   &   &   &   \\
Validación de instrumentos &   & X &   &   &   &   \\
Recolección de datos &   &   & X & X &   &   \\
Análisis de resultados &   &   &   & X & X &   \\
Redacción del informe &   &   &   &   & X & X \\
Correcciones finales y sustentación &   &   &   &   &   & X \\
\bottomrule
\end{tabular}
\begin{tablenotes}
\item Nota. El cronograma está organizado en función de los objetivos y fases metodológicas del estudio.
\end{tablenotes}
\end{threeparttable}
\end{table}

% ------------------- RECURSOS HUMANOS -------------------
\section{Recursos humanos}

El desarrollo de esta investigación contempla la participación de los siguientes recursos humanos:

\begin{table}[H]
\centering
\begin{threeparttable}
\caption{Recursos humanos involucrados en la investigación}
\label{tab:recursos_humanos}
\begin{tabular}{@{}p{4cm}p{9cm}@{}}
\toprule
\textbf{Rol} & \textbf{Descripción de funciones} \\
\midrule
Investigador principal & Responsable de todo el proceso investigativo, desde la formulación del problema hasta la elaboración del informe final. \\
Asesor académico & Brinda acompañamiento metodológico y técnico, asegurando el rigor científico del estudio. \\
Asistente de investigación (opcional) & Apoya en la recolección, codificación y sistematización de datos. \\
Jurado evaluador & Evalúa el trabajo final y la sustentación del proyecto de tesis. \\
\bottomrule
\end{tabular}
\begin{tablenotes}
\item Nota. Cada recurso cumple un rol clave para garantizar la calidad de la investigación.
\end{tablenotes}
\end{threeparttable}
\end{table}

% ------------------- RECURSOS MATERIALES -------------------
\section{Recursos materiales}

Los recursos materiales necesarios para el desarrollo del proyecto de investigación incluyen:

\begin{table}[H]
\centering
\begin{threeparttable}
\caption{Recursos materiales requeridos para la investigación}
\label{tab:recursos_materiales}
\begin{tabular}{@{}p{6cm}p{7cm}@{}}
\toprule
\textbf{Recurso} & \textbf{Utilidad en la investigación} \\
\midrule
Computadora portátil & Procesamiento de datos, redacción del informe y acceso a bibliografía digital. \\
Software de análisis (SPSS, Python, R, Weka) & Análisis estadístico y aplicación del algoritmo de árbol de decisión. \\
Acceso a internet & Revisión bibliográfica y uso de plataformas académicas. \\
Herramientas ofimáticas & Elaboración de gráficos, tablas, presentaciones y documentos. \\
Papelería y útiles & Organización y soporte físico de materiales. \\
Acceso a bases de datos científicas & Consulta de investigaciones previas y artículos especializados. \\
\bottomrule
\end{tabular}
\begin{tablenotes}
\item Nota. Se prioriza el uso de herramientas digitales de libre acceso o disponibles institucionalmente.
\end{tablenotes}
\end{threeparttable}
\end{table}

% ------------------- RECURSOS FINANCIEROS -------------------
\section{Recursos financieros (Presupuesto)}

A continuación, se detalla el presupuesto estimado requerido para la ejecución del presente estudio:

\begin{table}[H]
\centering
\begin{threeparttable}
\caption{Presupuesto estimado para la investigación}
\label{tab:presupuesto}
\begin{tabular}{@{}p{6.5cm}p{4.5cm}r@{}}
\toprule
\textbf{Concepto} & \textbf{Detalle} & \textbf{Monto (S/.)} \\
\midrule
Software especializado & Licencia de análisis estadístico o científico (SPSS, NVivo, Turnitin) & 500.00 \\
Impresión y empastado & Tres copias del trabajo final & 150.00 \\
Movilidad local & Visitas para encuestas o entrevistas (si aplica) & 200.00 \\
Papelería y útiles & Cuadernos, hojas, lapiceros, carpetas & 50.00 \\
Honorarios técnicos (opcional) & Asistencia en transcripción o análisis avanzado & 300.00 \\
\textbf{Total estimado} &  & \textbf{1,200.00} \\
\bottomrule
\end{tabular}
\begin{tablenotes}
\item Nota. El presupuesto puede variar según disponibilidad de recursos institucionales o colaboraciones académicas.
\end{tablenotes}
\end{threeparttable}
\end{table}


