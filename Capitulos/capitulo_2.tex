% ------------------- CAPÍTULO II. PLANTEAMIENTO DEL PROBLEMA -------------------

\section{Antecedentes}
En esta sección se presentan estudios previos nacionales e internacionales que abordan el uso del procesamiento de lenguaje natural en la atención al cliente, especialmente en el contexto de las pequeñas y medianas empresas del sector turismo.


\textcite{Akhtar}, ......


\textcite{Akhtar}, afirma que ``el modelo propuesto mejora la precisión del pronóstico.....''.

``El modelo propuesto mejora la precisión del pronóstico....'' \parencite[p. 45]{Bailey}.

\parencite{Bhandari1997} ...... ``...''

\section{Descripción y formulación del problema}
A pesar del crecimiento del sector turismo en el Perú, muchas PYMEs enfrentan dificultades para atender eficientemente a sus clientes debido a la limitada incorporación de tecnologías como la inteligencia artificial.  

\subsection{Problema general}
¿Cómo influye un modelo de procesamiento de lenguaje natural para mejorar la atención al cliente en PYMEs del sector turismo, Tacna 2025?

\subsection{Problemas específicos}
¿Cómo influye un modelo de procesamiento de lenguaje natural para mejorar la capacidad de respuesta de atención al cliente en PYMEs del sector turismo, Tacna 2025.

¿Cómo influye un modelo de procesamiento de lenguaje natural para mejorar la confiabilidad de atención al cliente en PYMEs del sector turismo, Tacna 2025.

¿Cómo influye un modelo de procesamiento de lenguaje natural para mejorar la credibilidad de atención al cliente en PYMEs del sector turismo, Tacna 2025.

\section{Objetivos}
\subsection{Objetivo general}
Implementar un modelo de procesamiento de lenguaje natural para mejorar la atención al cliente en PYMEs del sector turismo, Tacna 2025.

\subsection{Objetivos específicos}
Diseñar un modelo de procesamiento de lenguaje natural para mejorar la capacidad de respuesta de atención al cliente en PYMEs del sector turismo, Tacna 2025.

Desarrollar un modelo de procesamiento de lenguaje natural para mejorar la confiabilidad de atención al cliente en PYMEs del sector turismo, Tacna 2025.

Evaluar un modelo de procesamiento de lenguaje natural para mejorar la credibilidad de atención al cliente en PYMEs del sector turismo, Tacna 2025.

\section{Justificación e importancia de la investigación}
Se realiza como mínimo en un párrafo por cada uno de estos tipos, justificación teórica, justificación metodológica, justificación práctica, justificación tecnológica.

\section{Limitaciones y viabilidad}
\textbf{Limitaciones}

Se escribe sobre las posibles limitaciones sin que estos afecten el desarrollo de la investigación.

\textbf{Viabilidad}

Escribir al menos un párrafo sobre la viabilidad económica, viabilidad tecnológica, viabilidad operativa.

\section{Hipótesis}
\textbf{Hipótesis general}

Implementar el modelo de procesamiento de lenguaje natural influye positivamente para mejorar la atención al cliente en PYMEs del sector turismo, Tacna 2025.
\\

\textbf{Hipótesis específica}

\begin{itemize}
    \item Diseñar un modelo de procesamiento de lenguaje natural influye positivamente para mejorar la capacidad de respuesta de atención al cliente en PYMEs del sector turismo, Tacna 2025.
    \item Desarrollar un modelo de procesamiento de lenguaje influye positivamente natural para mejorar la confiabilidad de atención al cliente en PYMEs del sector turismo, Tacna 2025.
    \item Evaluar un modelo de procesamiento de lenguaje natural influye positivamente para mejorar la credibilidad de atención al cliente en PYMEs del sector turismo, Tacna 2025.
\end{itemize}

\section{Variables}

\subsection{Definición conceptual y operacional de las variables}
\textbf{
Variable independiente Modelo de procesamiento de lenguaje natural
}
\begin{itemize}
    \item Definición conceptual: 
    \item Definición operacional:
\end{itemize}

\textbf{
Variable dependiente atención al cliente
}
\begin{itemize}
    \item Definición conceptual: 
    \item Definición operacional: 
\end{itemize}

\subsection{Operacionalización de las variables}

\begin{table}[H] % La opción H fuerza la tabla a estar exactamente donde la pones
\centering
\caption{Operacionalización de variables}
\label{tab:operacionalizacion}
\begin{tabular}{p{3.2cm} p{3cm} p{3.5cm} p{1cm} p{3cm}} % No hay líneas verticales
\toprule
\textbf{Variable} & \textbf{Dimensiones} & \textbf{Indicadores} & \textbf{Ítem} & \multicolumn{1}{c}{\textbf{Escala}} \\
\midrule
Modelo de procesamiento de lenguaje natural & Procesamiento sintáctico & Identificación de estructuras gramaticales & 1 & \multirow{3}{*}{Ordinal tipo Likert} \\
 & Procesamiento semántico & Comprensión del contexto semántico & 2 & \\
 & Reconocimiento de entidades & Identificación de entidades nombradas & 3 & \\
\midrule
Atención al cliente & Rapidez de respuesta & Tiempo de respuesta & 4 & \multirow{3}{*}{Ordinal tipo Likert} \\
 & Calidad de la respuesta & Claridad y precisión de la respuesta & 5 & \\
 & Satisfacción del cliente & Nivel de satisfacción general & 6 & \\
\bottomrule
\end{tabular}
\caption*{\textit{Nota}. \normalfont Desarrollado en base a la teoría.}
\end{table}

La tabla \ref{tab:operacionalizacion2} contiene datos de ejemplo.

\begin{table}[H] % La opción H fuerza la tabla a estar exactamente donde la pones
\centering
\caption{Operacionalización de variables}
\label{tab:operacionalizacion2}
\begin{tabular}{|p{3.2cm}|p{3cm}|p{3.5cm}|p{1cm}|p{3cm}|}
\hline
\textbf{Variable} & \textbf{Dimensiones} & \textbf{Indicadores} & \textbf{Ítem} & \multicolumn{1}{c|}{\textbf{Escala}} \\
\hline
Modelo de procesamiento de lenguaje natural & Procesamiento sintáctico & Identificación de estructuras gramaticales & 1 & \multirow{3}{*}{Ordinal tipo Likert} \\
 & Procesamiento semántico & Comprensión del contexto semántico & 2 & \\
 & Reconocimiento de entidades & Identificación de entidades nombradas & 3 & \\
\hline
Atención al cliente & Rapidez de respuesta & Tiempo de respuesta & 4 & \multirow{3}{*}{Ordinal tipo Likert} \\
 & Calidad de la respuesta & Claridad y precisión de la respuesta & 5 & \\
 & Satisfacción del cliente & Nivel de satisfacción general & 6 & \\
\hline
\end{tabular}
\caption*{\textit{Nota}. \normalfont Desarrollado en base a la teoría.}
\end{table}

Texto después de la tabla, el cual continuará de forma normal.



