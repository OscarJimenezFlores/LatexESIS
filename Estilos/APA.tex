%-----------------------------------------------------
% Dr. Oscar Jimenez Flores
% Configuración de APA y formato de índice alineado
%-----------------------------------------------------

% Configuración del tamaño de página
\usepackage[a4paper,
            left=3.5cm,
            right=2.5cm,
            top=2.5cm,
            bottom=2.5cm]{geometry}

% Idioma y codificación
\usepackage[spanish]{babel}
\usepackage[utf8]{inputenc}
\usepackage[autostyle=true, babel=true]{csquotes}

% Tipografía y espaciado APA
\usepackage{times} %Times New Roman
%\usepackage{newtxtext}
\renewcommand{\normalsize}{\fontsize{12}{15}\selectfont} % Aquí ajustas el tamaño de fuente a 12pt
\usepackage{setspace}
\onehalfspacing
\setlength{\parindent}{0.8cm} % sangría de la primera línea de párrafo
\usepackage{indentfirst}

% Matemáticas
\usepackage{amsmath}

% Figuras y tablas
\usepackage{graphicx}
%\graphicspath{Figuras/}
\usepackage{caption}
\usepackage{array}

% Bibliografía estilo APA
\usepackage[backend=biber, style=apa, sorting=nyt]{biblatex}
\addbibresource{Bibliografia/referencias.bib}
\DeclareLanguageMapping{spanish}{spanish-apa}

% Hipervínculos
\usepackage{url}
\usepackage{hyperref}
\hypersetup{
    colorlinks=true,
    linkcolor=black,
    filecolor=magenta,      
    urlcolor=black,
    citecolor=black,
}
\urlstyle{same}

% Numeración de páginas
\pagestyle{plain}

% Eliminar páginas en blanco entre capítulos
\let\cleardoublepage\clearpage

% Numeración romana para capítulos
%\renewcommand{\thechapter}{\Roman{chapter}}

% Paquetes para formato de títulos e índice
\usepackage{titlesec}
\usepackage{titletoc}

% ---------------------- FORMATO DE TÍTULOS ----------------------

% Formato para capítulos
\titleformat{\chapter}[display]
  {\normalfont\bfseries\centering\normalsize}
  {CAPÍTULO \ \Roman{chapter}} % Primera línea
  {0pt}
  {\vspace{1ex}\centering} % Segunda línea será el título real
\titlespacing*{\chapter}{0pt}{0pt}{20pt}

% Formato para secciones y subsecciones
\titleformat{\section}
  {\normalfont\normalsize\bfseries}
  {\thesection}
  {0.3em} % espacio entre numeración y título
  {}
\titlespacing*{\section}{0pt}{12pt}{6pt}

\titleformat{\subsection}
  {\normalfont\normalsize\bfseries}
  {\thesubsection}
  {0.3em}
  {}
\titlespacing*{\subsection}{0pt}{12pt}{6pt}

\titleformat{\subsubsection}
  {\normalfont\normalsize\bfseries}
  {\thesubsubsection}
  {0.3em}
  {}
\titlespacing*{\subsubsection}{0pt}{12pt}{6pt}

%\renewcommand{\thechapter}{\Roman{chapter}}
\renewcommand\thesection{\thechapter.\arabic{section}.}
\renewcommand\thesubsection{\thesection\arabic{subsection}.}
\renewcommand\thesubsubsection{\thesubsection\arabic{subsubsection}.}

%\setcounter{secnumdepth}{3}
%\setcounter{tocdepth}{3}


% ---------------------- CONFIGURACIÓN DEL ÍNDICE ----------------------

% Capítulos en índice: "I. Datos generales"
\titlecontents{chapter}
  [0pt] % Alineado al margen izquierdo del documento
  {\addvspace{4pt}\normalfont\normalsize}
  {\thecontentslabel. \enspace} % I. Nombre capítulo
  {} % No hay texto antes del título
  {\titlerule*[0.5pc]{.}\contentspage}

% Secciones alineadas al mismo margen
\titlecontents{section}
  [0pt]
  {\addvspace{2pt}\normalfont\normalsize}
  {\thecontentslabel \enspace}
  {}
  {\titlerule*[0.5pc]{.}\contentspage}

% Subsecciones también alineadas
\titlecontents{subsection}
  [0pt]
  {\addvspace{2pt}\normalfont\normalsize}
  {\thecontentslabel \enspace}
  {}
  {\titlerule*[0.5pc]{.}\contentspage}

% Incluir palabra "bibliografía" en el índice automáticamente
%\usepackage[nottoc]{tocbibind}

% ------------------- Comando para Título de ANEXOS -------------------
\newcommand{\anexosTitulo}[1]{%
    \cleardoublepage
    \phantomsection
    \addcontentsline{toc}{chapter}{#1}
    \thispagestyle{empty}
    %\vspace*{\fill}
    \vspace*{8.5cm} % ↑ Ajusta este valor según qué tan arriba lo quieras
    \begin{center}
        {\fontsize{12}{15}\selectfont\bfseries #1}
    \end{center}
    \vspace*{\fill}
}

% ---------------------- CONFIGURACIÓN DEL PÁGINAS ANEXOS ----------------------
\usepackage{pdflscape}

% Macro para anexos verticales
\newcommand{\anexoV}[2]{ % #1 = título del anexo, #2 = archivo
    \newpage
    \cleardoublepage
    \phantomsection
    %\addcontentsline{toc}{chapter}{#1}  % Agregar el título del anexo al índice
    \input{#2}  % Cargar el archivo en formato vertical
    \thispagestyle{empty}  % Eliminar numeración en esta página

}

% Macro para anexos horizontales
\newcommand{\anexoH}[2]{ % #1 = título del anexo, #2 = archivo
    \newpage
    \phantomsection
    %\addcontentsline{toc}{chapter}{#1}  % Agregar título al índice
    
    \newgeometry{top=1cm, left=1cm, right=1cm, bottom=1cm}  % Ajuste de márgenes
    
    \begin{landscape}  % Inicia el entorno horizontal
        \thispagestyle{empty}  % Eliminar numeración de página
        \input{#2}  % Cargar el archivo
        \thispagestyle{empty}  % Eliminar numeración de página
    \end{landscape}
    %\thispagestyle{empty}  % Eliminar numeración de página
    \restoregeometry  % Restaurar los márgenes originales
}

% ---------------------- CONFIGURACIÓN TABLAS ----------------------
\usepackage{booktabs}
\usepackage{float}
\usepackage{threeparttable}
\usepackage{longtable}
\usepackage{multirow}

\counterwithout{table}{chapter}
\counterwithout{table}{section}% Desactiva numeración tipo 2.1 → deja solo 1, 2, 3
\addto\captionsspanish{\renewcommand{\tablename}{Tabla}}
\DeclareCaptionFormat{apaespanol}{%
  \normalfont\bfseries#1 #2\\
  \normalfont\itshape#3
}
\captionsetup[table]{format=apaespanol, labelsep=none, textfont=it, justification=raggedright, singlelinecheck=off}
\captionsetup[table]{justification=raggedright, singlelinecheck=off, labelsep=none} % Alinea el título a la izquierda

% ---------------------- CONFIGURACIÓN FIGURAS ----------------------
% Configuración para que las figuras sigan el formato APA
\captionsetup[figure]{
  format=apaespanol, 
  labelsep=none, 
  textfont=it, 
  justification=raggedright, 
  singlelinecheck=off, 
  font=small
}
\addto\captionsspanish{\renewcommand{\figurename}{Figura}} % Cambia "Figure" por "Figura"

% Ajuste de numeración para figuras
\counterwithout{figure}{chapter}  % Elimina la numeración por capítulo, para numeración continua
\counterwithout{figure}{section}  % Elimina la numeración por sección, para numeración continua

% Ajuste de estilo para la numeración de las figuras
\DeclareCaptionFormat{apaespanol}{%
  \normalfont\bfseries#1 #2\\
  \normalfont\itshape#3
}